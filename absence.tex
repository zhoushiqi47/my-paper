
\documentclass[12pt]{iopart}
%\pdfoutput=1
\usepackage{iopams}
\usepackage{amssymb, epsfig}
%\usepackage{amsmath, amssymb,epsfig}
\usepackage{latexsym}

%\usepackage[hypertex,hyperindex]{hyperref}
\usepackage{showkeys}
\usepackage{graphicx}
\usepackage{color}

\newcommand{\pf}{\mbox{pf}}

\begin{document}
	
	\bibliographystyle{plain}
	\def\debproof{\noindent {\bf Proof.} }
	\def\finproof{\hfill {\small $\Box$} \\}
	%\renewcommand{\theequation}{\arabic{section}.\arabic{equation}}
	%\tableofcontents
	\makeatletter % `@' now normal "letter"
	\@addtoreset{equation}{section}
	\makeatother  % `@' is restored as "non-letter"
	\renewcommand\theequation{{\thesection}.{\arabic{equation}}}
	
\title[]{Absense of Positive Eigenvalues for the Linearized Elasticity System in the Half Space}
\author{ Zhiming Chen, Shiqi Zhou }
\address{LSEC, Institute of Computational Mathematics, Academy of
	Mathematics and Systems Science, Chinese Academy of Sciences,
	Beijing 100190, China}

\begin{abstract}
	In this paper, we prove that the linearized elasticity system in the half-space with traction free boundry has no eigenvalues. 
\end{abstract}
	\maketitle
	\newcommand{\eps}{\varepsilon}
	\newcommand{\RR}{\mathcal{R}}
	\newtheorem{lem}{Lemma}[section]
	\newtheorem{prop}{Proposition}[section]
	\newtheorem{cor}{Corollary}[section]
	\newtheorem{thm}{Theorem}[section]
	\newtheorem{rem}{Remark}[section]
	\newtheorem{alg}{Algorithm}[section]
	\newtheorem{assum}{Assumption}[section]
	\newtheorem{definition}{Definition}[section]
	
	
	\newcounter{RomanNumber}
	\newcommand{\MyRoman}[1]{\rm\setcounter{RomanNumber}{#1}\Roman{RomanNumber}}
	
	\newcommand{\bL}{\mathbf{L}}
	\newcommand{\bH}{\mathbf{H}}
	\newcommand{\bW}{\mathbf{W}}
	\newcommand{\bP}{\mathbf{P}}
	\newcommand{\bQ}{\mathbf{Q}}
	\newcommand{\bp}{\mathbf{p}}
	\newcommand{\bq}{\mathbf{q}}
	\newcommand{\uL}{u_{_{\rm L}}}
	\newcommand{\vL}{v_{_{\rm L}}}
	\newcommand{\tuL}{\tilde u_{_{\rm L}}}
	\newcommand{\tvL}{\tilde v_{_{\rm L}}}
	\newcommand{\fL}{f_{_{\rm L}}}
	\newcommand{\gL}{g_{_{\rm L}}}
	\newcommand{\bpL}{\bp_{_{\rm L}}}
	\newcommand{\bqL}{\bq_{_{\rm L}}}
	\newcommand{\tbpL}{\tilde{\bp}_{_{\rm L}}}
	\newcommand{\tbqL}{\tilde{\bq}_{_{\rm L}}}
	\newcommand{\tbpLf}{\tilde{\bp}_{_{\rm L,1}}}
	\newcommand{\tbpLs}{\tilde{\bp}_{_{\rm L,2}}}
	\newcommand{\tbqLf}{\tilde{\bq}_{_{\rm L,1}}}
	\newcommand{\tbqLs}{\tilde{\bq}_{_{\rm L,2}}}
	\newcommand{\bn}{\nu}
	\newcommand{\bv}{\mathbf{v}}
	\newcommand{\om}{\omega}
	\newcommand{\pa}{\partial}
	\newcommand{\la}{\langle}
	\newcommand{\ra}{\rangle}
	\newcommand{\lla}{\la{\hskip -2pt}\la}
	\newcommand{\rra}{\ra{\hskip -2pt}\ra}
	\newcommand{\jj}{\|{\hskip -0.8pt} |}
	\newcommand{\al}{\alpha}
	\newcommand{\ze}{\zeta}
	\newcommand{\si}{\sigma}
	\newcommand{\ep}{\varepsilon}
	\newcommand{\na}{\nabla}
	\newcommand{\vp}{\varphi}
	\newcommand{\ga}{\gamma}
	\newcommand{\Ga}{\Gamma}
	\newcommand{\Om}{\Omega}
	\newcommand{\de}{\delta}
	\newcommand{\Th}{\Theta}
	\newcommand{\De}{\Delta}
	\newcommand{\Lam}{\Lambda}
	\newcommand{\lam}{\lambda}
	\newcommand{\tri}{\triangle}
	\newcommand{\lj}{[{\hskip -2pt} [}
	\newcommand{\rj}{]{\hskip -2pt} ]}
	\newcommand{\bks}{\backslash}
	%\newcommand{\diag}{\mathrm{diag}}
	\newcommand{\diam}{\mathrm{diam}}
	\newcommand{\osc}{\mathrm{osc}}
	\newcommand{\meas}{\mathrm{meas}}
	\newcommand{\dist}{\mathrm{dist}}
	
	\newcommand{\mL}{\mathscr{L}}
	\newcommand{\cT}{{\cal T}}
	\newcommand{\cM}{{\cal M}}
	\newcommand{\cE}{{\cal E}}
	\newcommand{\cL}{{\cal L}}
	\newcommand{\cF}{{\cal F}}
	\newcommand{\cB}{{\cal B}}
	\newcommand{\PML}{{\rm PML}}
	\newcommand{\FEM}{{\rm FEM}}
	\newcommand{\rd}{\,\mathrm{d}}
	
	\renewcommand{\i}{\mathbf{i}}
	\renewcommand{\v}{\mathbf{v}}
	\renewcommand{\u}{\mathbf{u}}

	\newcommand{\U}{\mathbf{U}}
	\newcommand{\w}{\mathbf{w}}
	\newcommand{\q}{\mathbf{q}}
	\newcommand{\W}{\mathbf{W}}
	\newcommand{\Q}{\mathbf{Q}}



\newcommand{\A}{{\mathbb{A}}}
	\renewcommand{\r}{\mathbf{r}}
	\newcommand{\R}{{\mathbb{R}}}
	\newcommand{\Z}{{\mathbb{Z}}}
	\newcommand{\C}{{\mathbb{C}}}
	\newcommand{\I}{{\mathbb{I}}}
	\renewcommand{\Re}{\mathrm{Re}\,}
	\renewcommand{\Im}{\mathrm{Im}\,}
	\renewcommand{\div}{\mathrm{div}}
	\newcommand{\curl}{\mathrm{curl}}
	\newcommand{\Curl}{\mathbf{curl}}
	
	\newcommand{\Np}{\mathcal{N}_p}
	\newcommand{\Ns}{\mathcal{N}_s}
	\newcommand{\Tp}{\mathcal{T}_p}
	\newcommand{\Ts}{\mathcal{T}_s}
	
	%%%%%%%%%%%%%%%%%%%%%%%%%%%%%%%%%%%%%%%%%%%%%%%%%%%%%%%%%%%%%%%%%%%%
	\newcommand{\be}{\begin{eqnarray}}
		\newcommand{\ee}{\end{eqnarray}}
	\newcommand{\ben}{\begin{eqnarray*}}
		\newcommand{\een}{\end{eqnarray*}}
	\newcommand{\nn}{\nonumber}
	

	
	\section{Introduction}\label{section1}
	In this paper, we consider the linearized and isotropic elasticity system defined on an unbounded domain $\Omega=\R_+^2\bks \bar{D}$ with traction free surface $\Ga_0=\{(x_1,x_2)^T:x_1\in\R,x_2=0\}$, where $D\subsetneq\R^2_+$ is a bounded Lipschitz domain with the unit outer normal $\nu$ to its boundary $\Gamma_D$.
	We study the eigenvalues of the  following elastic scattering problem in the isotropic homogeneous medium half space with \emph{Lam\'{e}} constant $\lambda$ and $\mu$ and constant density $\rho\equiv1$:
	\be\label{elastic_eq}
	\nabla\cdot\sigma(\u) + \rho\omega^2 \u= f \ \ \ \ &\mbox{in }& \R_+^2\bks \bar{D}\\ \label{elastic_bd}
	\u=0 \ \ \mbox{on} \ \Ga_D  \ \ \mbox{and} \ \ \sigma(\u)\cdot e_2=0 \  \ \ \ &\mbox{on}& \ \Ga_0
	\ee
	together with the constitutive relation (Hookes law)
	\ben
	\sigma(\u) = 2\mu\ep(\u) + \lambda\div \u \I \\
	\ep(\u)=\frac{1}{2}(\na \u +(\na \u)^T)
	\een
	where $\omega$ is the circular frequency, $\u(x_1,x_2)=(u_1(x),u_2(x))^T\in\C^2$ denotes the displacement fields and $\sigma(u)$ is the stress tensor. We also need to define the surface traction $T_x^n (\cdot)$ on the normal direction n,
	\ben
	T_x^n \u(x) := \sigma\cdot n = 2\mu\frac{\pa \u}{\pa n}+\lambda n\div \u + \mu n \times \curl \u
	\een
	For simplicity, let's introduce \emph{Lam\'{e}} operator $\Delta_e$ as
	\ben
	\Delta_e \u = (\lambda+2\mu)\nabla\nabla\cdot \u - \mu\nabla\times\nabla\times u=\nabla\cdot\sigma(\u)
	\een
	We remark that the results in
	this paper can be extended to other boundary
	conditions such as Neumann or mixed boundary conditions on $\Ga_D$, or even to penetrable obstacle.
	
	In order to complete the definition of the proble, we introduce the domain of the operater $\De_e$
	\ben
	\mathcal{D}(\De_e,S)=\{v\in H^1(S): \De_e v\in L^2(S), \sigma(v)e_2=0\ \ \mbox{on }\Ga_0\}
	\een
	where $S$ is an unbounded domain in $\R^2_+$.
	For the elasticity system, the study of eigenvalue is little.  
	The layout of the paper is as follows. In section 2 
	
	\section{Absence of Positive Eigenvalues}
	In this section. Throughout the paper, we will assume that for $z\in\mathbb{C}$, $z^{1/2}$ is the analytic branch of $\sqrt{z}$ such that $\Im (z^{1/2})\geq0$. This corresponds to the rigt half real axis as the branch cut in the complex plane. For $z=z_1+\i z_2,z_1,z_2\in\mathbb{R}$, we have
	\be \label{convention_1}
	z^{1/2}=sgn(z_2)\sqrt{\frac{|z|+z_1}{2}}+\i\sqrt{\frac{|z|-z_1}{2}}
	\ee
	For $z$ on the right half real axis, we take $z^{1/2}$ as the limit of $(z+\i\ep)^{1/2}$ as $\ep \to 0^+$.
	\begin{thm}\label{2.1}
		Let $\om\in \R$ and $\u$ statisfy the equations (\ref{elastic_eq}-\ref{elastic_bd}) in $\mathcal{D}(\De_e,\Om)$,
		then we assert $\u=0$.
	\end{thm}
	\begin{lem} \label{lem2.1}
		The Rayleigh equation $\delta(\xi) = 0$ has only two zeros $\pm k_R$, $k_R>k_s$, in the complex plane. 
	\end{lem}
	
	\begin{lem}\label{lem2.2}
		Suppose that $f\in L^2(\R^2_+)$ with compact support in $B\subsetneq \R\times(h,+\infty), \ h>0$. Let $\omega\in \R$ and $\u\in \mathcal{D}(\De_e,\R^2_+)$ such that:
		\be\label{eq2}
		\De_e \u +\om^2\u = f
		\ee
		then we assert $\u=0$ in $(\R\times(h,+\infty))$.
	\end{lem}
	\debproof
	Let $\mathcal{F}_{x_1}(\cdot): L^2(R^2_+) \rightarrow L^2(R^2_+)$ be the partial Fourier tranfor given by $\hat{g}:=\mathcal{F}_{x_1}(g):=\int_\R g(x_1,x_2)e^{\i x_1\xi}dx_1$.
	By taking the Fourier transform of (\ref{eq2}) and (\ref{elastic_bd}), we obtain ODEs for $x_2$ in $R_+$
	\be
	& & \mu \frac{d^2\hat u_1}{dx_2^2}+\i(\lambda+\mu)\xi\frac{d\hat u_2}{dx_2}+(\omega^2-(\lambda+2\mu)\xi^2)\hat u_1 =\hat{f_1} \label{pp3}\\
	& & (\lambda+2 \mu)\frac{d^2\hat u_2}{dx_2^2}+\i(\lambda+\mu)\xi\frac{d\hat u_1}{dx_2}+(\omega^2-\mu \xi^2)\hat u_2 = \hat{f_2} \label{pp4}
	\ee
	and the boundary coditions on $x_2=0$ are
	\be
	& & \mu\frac{d\hat u_1}{dx_2}+\i\mu\xi\hat u_2 = 0\label{pp5}\\
	& & (\lambda+2\mu)\frac{d\hat u_2}{dx_2}+\i\lambda\xi\hat u_1 = 0 \label{pp6}
	\ee
	In order to work with real coefficient, we use the change of variables:
	\ben
	v_1=\i \hat{u}_1 , \ \ \ \  v_2=\hat{u}_2,  \ \ \ \  \v=(v_1,v_2)^T \\
	q_1=\i \hat{f}_1 , \ \ \ \  q_2=\hat{f}_2,  \ \ \ \  \q=(q_1,q_2)^T
	\een
	Then we have the following equations:
	\be\label{eq3}
	[\A_1 \frac{d^2}{dx_2^2} +(\A_2-(\A_2)^T)\xi\frac{d}{dx_2}-\A_3\xi^2+\omega^2]\v=\q \ \ \ \ \ &\mbox{in}& \ \  \R_+ \\
	(\A_1 \frac{d}{dx_2} +\A_2\xi) \ \ \v =0 \ \ \ &\mbox{on}& \ \ x_2=0
	\ee
	where
	\ben\hspace{-1cm}
	\A_1=\Bigg( \begin{array}{cc}
		\mu &  0\\
		0 & \lambda+2\mu
	\end{array} \Bigg) ,\ \  \ \ \
	\A_2=\Bigg( \begin{array}{cc}
		0 & -\mu\\
		\lambda & 0
	\end{array} \Bigg), \ \ \ \
	\A_3=\Bigg( \begin{array}{cc}
		\lambda+2\mu &  0\\
		0 & \mu
	\end{array} \Bigg)
	\een
	Let $\w$ be the solution of the following equations:
	
	
	\ben
	[\A_1 \frac{d^2}{dx_2^2} +(\A_2-(\A_2)^T)\xi\frac{d}{dx_2}-\A_3\xi^2+\omega^2]\w=\q \ \ \ \ \ &\mbox{in}& \ \  (0,h) \\
	\w=0, \ \ \  (\A_1 \frac{d}{dx_2} +\A_2\xi) \ \ \w =0 \ \ \ &\mbox{on}& \ \ x_2=h
	\een
	It is easy to transform above equations into a simpler form by variables substitution $\W=(\w,(\A_1 \frac{d}{dx_2} +\A_2\xi) \w)^T$, $\Q=(0,0,\q)^T$
	\ben
	\frac{d}{dx_2}\W = \A \W +\Q \ \ \ \ \ \mbox{in} \ \  (0,h) \\
	\W=0  \ \ \ \mbox{on} \ \ x_2=h
	\een
	where 
	\ben
	\A=\left(
	\begin{array}{cc}
		-\A_1^{-1}\A_2\xi & \A_1^{-1} \\
		-\A_2^T\A_1^{-1}\A_2\xi^2+\A_3\xi^2-\om^2 & \A_2^T\A_1^{-1}\xi
	\end{array}
	\right)
	\een
	By the standard arguments in ODEs, we can obtain
	\ben
	\W(\xi,x_2)=-\Phi(\xi,x_2) \int_{h}^{x_2} \Phi^{-1}(\xi,t)\Q(\xi,t)dt
	\een
	where
	\ben\hspace{-1.5cm}
	\Phi(\xi,t)=\left(\begin{array}{cccc}
		-\mu_s(\xi) e^{\i \mu_s t} & -\xi e^{\i\mu_p(\xi) t} & -\mu_s(\xi) e^{-\i \mu_s(\xi) t} & \xi e^{-\i\mu_p t}  \\
		-\i \xi e^{\i \mu_s(\xi) t} & \i\mu_p(\xi) e^{\i\mu_p(\xi) t} & \i \xi e^{-\i \mu_s(\xi) t} & \i\mu_p(\xi)  e^{-\i\mu_p(\xi) t} \\
		-\i\mu\beta(\xi) e^{\i \mu_s t} & -2\i\mu \xi\mu_p(\xi)  e^{\i\mu_p(\xi) t} & \i\mu\beta(\xi) e^{-\i \mu_s(\xi) t} & -2\i\mu \xi\mu_p(\xi)  e^{-\i\mu_p t} \\
		2\mu \xi \mu_s(\xi) e^{\i \mu_s t} & -\mu \beta(\xi)  e^{\i\mu_p(\xi) t} & 2\mu \xi \mu_s(\xi) e^{-\i \mu_s t} & \mu \beta(\xi)   e^{-\i\mu_p(\xi) t}
	\end{array}\right)
	\een
	Here $k_p=\omega/\sqrt{\lam+2\mu}, k_s=\omega/\sqrt{\mu}$ are wave number of p-wave and s-wave, and $\mu_\alpha=(k_\alpha^2-\xi^2)^{1/2}$ for $\alpha=s,p$.
	
	We extend $\w(\xi,x_2)$ by zero in $(h,\infty)$. Therefore. $\w(\xi,x_2)$ satisfy equation \ref{eq3} in $\R_+$.
	Since $\Phi(\xi,t)$ are analytic w.r.t $\xi$ in $\R\bks\{k_p,k_s\}$ and f(x) have compact support, we deduce that for almost every $\xi\in \R$, $\w(\xi,x_2)$ are analytic and so $(\A_1 \frac{d}{dx_2} +\A_2\xi) \w$ are.
	
	
	
	We set $\U=\v-\w$ and $\U=(U_1,U_2)^T$. Then $\U$ satify the following Cauchy problem:
	\be\label{eq4}
	[\A_1 \frac{d^2}{dx_2^2} +(\A_2-(\A_2)^T)\xi\frac{d}{dx_2}-\A_3\xi^2+\omega^2]\U=0 \ \ \ \ \ &\mbox{in}& \ \  \R_+ \\
	(\A_1 \frac{d}{dx_2} +\A_2\xi)  \U =(\A_1 \frac{d}{dx_2} +\A_2\xi)   \w \ \ \ &\mbox{on}& \ \ x_2=0
	\ee
	Since the coefficients of above equations are constants, we can represent $\U(\xi,x_2)$ in the following form:
	\ben\hspace{-2.5cm}
	\U(\xi,x_2)=c_1(\xi)\left(\begin{array}{l}
		-\mu_s \\
		-\i \xi
	\end{array}\right)e^{\i \mu_s x_2}+c_2(\xi)\left(\begin{array}{l}
		-\xi \\
		\i \mu_p
	\end{array}\right)e^{\i \mu_p x_2}+c_3(\xi)\left(\begin{array}{l}
		-\mu_s \\
		\i \xi
	\end{array}\right)e^{-\i \mu_s x_2}+c_4(\xi)\left(\begin{array}{l}
		\xi \\
		\i \mu_p
	\end{array}\right)e^{-\i \mu_p x_2}
	\een
	
	If $\xi^2\leq k^2_p$, then it's simple to see that $\U=0$ in $L^2_{x_2}(\R_+)$. So, for $\xi^2<k^2_p$, $(\A_1 \frac{d}{dx_2} +\A_2\xi)  \U =0$ which implis $(\A_1 \frac{d}{dx_2} +\A_2\xi)   \w =0$. Since $(\A_1 \frac{d}{dx_2} +\A_2\xi)   \w$ are analytic for almost every $\xi \in \R$, we deduce that 
	\be\label{bd_1}
	(\A_1 \frac{d}{dx_2} +\A_2\xi)  \U =0  \ \ \ \ \ \mbox{on} \ \ \ x_2=0
	\ee
	for almost every $\xi \in \R$. Therefore, we can obtain
	
	\ben
	\U(\xi,x_2) = \left\{
	\begin{array}{l}
		c(\xi)\left(\begin{array}{l}
			-\xi \\
			\i \mu_p
		\end{array}\right)e^{\i \mu_p x_2},     \ \ \ \ \   k_p^2<   \xi^2\leq k_s^2  \\
		c_1(\xi)\left(\begin{array}{l}
			-\mu_s \\
			-\i \xi
		\end{array}\right)e^{\i \mu_s x_2}+c_2(\xi)\left(\begin{array}{l} 
			-\xi \\
			\i \mu_p
		\end{array}\right)e^{\i \mu_p x_2}, \ \ \xi^2>k_s^2
	\end{array}
	\right.
	\een
	\ben \hspace{-2cm}
	(\A_1 \frac{d}{dx_2} +\A_2\xi)  \U = \left\{
	\begin{array}{l}
		c(\xi)\left(\begin{array}{l}
			-2\i\mu\xi\mu_p \\
			-\mu\beta
		\end{array}\right)e^{\i \mu_p x_2},     \ \ \ \ \   k_p^2<   \xi^2\leq k_s^2  \\
		c_1(\xi)\left(\begin{array}{l}
			-\i\mu\beta \\
			2\mu\xi\mu_s
		\end{array}\right)e^{\i \mu_s x_2}+c_2(\xi)\left(\begin{array}{l} 
			-2\i\mu\xi\mu_p \\
			-\mu\beta
		\end{array}\right)e^{\i \mu_p x_2}, \ \ \xi^2>k_s^2
	\end{array}
	\right.
	\een
	By boudary condition \ref{bd_1}, we have $c(\xi)=0$ for $k_p^2<   \xi^2\leq k_s^2$ and 
	
	\be\hspace{-2cm}
	\mathbf{det} \left(\begin{array}{ll}
		-\i\mu\beta & -2\i\mu\xi\mu_p \\
		2\mu\xi\mu_s &	-\mu\beta
	\end{array}\right)=-\i\mu(\beta^2+4\xi^2\mu_s\mu_p)=0
	\ \ \ \  \mbox{for} \ \ \xi^2>k_s^2
	\ee
	Therefore, by lemma \ref{lem2.1} we have $\U(\xi,x_2)=0$ for almost every $\xi\in\R$ which implis $\v(\xi,x_2)=0$  for almost every $\xi\in\R$ and $x_2\in(h,+\infty��$. This completes the proof by taking the inverse Fourier tranformation of $\hat{\u}(\xi,x_2)$. 
	\finproof
	
	{\bf proof of Theorem \ref{2.1}:} Since $D\subsetneq\R^2_+$,we can find two concentric circles $B_{R_1},B_{R_2}$ such that $D\subsetneq B_{R_1}\subsetneq B_{R_2}  \subsetneq \R^2_+$. Let $\chi \in C_{0}^{\infty}(\R^2_+)$ be the cut-off function such that $0 \leq \chi \leq 1$, $\chi=0$ in $B_{R_1}$, and $\chi=1$ outside of $B_{R_2}$.
	Let $v=\chi u$.
	Then $v$ satisfies (\ref{eq2}) with
	$f=\sigma(u)\na\chi+(\lambda+\mu)(\na^2 \chi u+ \na u \na\chi)+\mu\Delta\chi u +\mu\div u\na\chi$, where $\na^2 \chi$ is the Hessian matrix of $\chi$. Clearly $q$ has compact support. By lemme \ref{lem2.2}, we have $u=v=0$ in $\R\times(h,+\infty)$. Finally, the unique continuation principle
	implies that u=0 in $\R^2_+$. This completes the proof. \cite{Ahlfors1979Complex}
	\finproof
	

	
	\section*{References}
	\bibliography{eee}
\end{document}