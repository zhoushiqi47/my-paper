
\documentclass[12pt]{iopart}
%\pdfoutput=1
\usepackage{iopams}
\usepackage{amssymb, epsfig}
%\usepackage{amsmath, amssymb,epsfig}
\usepackage{latexsym}

%\usepackage[hypertex,hyperindex]{hyperref}
\usepackage{showkeys}
\usepackage{graphicx}
\usepackage{color}

\newcommand{\pf}{\mbox{pf}}

\begin{document}

\bibliographystyle{plain}
\def\debproof{\noindent {\bf Proof.} }
\def\finproof{\hfill {\small $\Box$} \\}
%\renewcommand{\theequation}{\arabic{section}.\arabic{equation}}
%\tableofcontents
\makeatletter % `@' now normal "letter"
\@addtoreset{equation}{section}
\makeatother  % `@' is restored as "non-letter"
\renewcommand\theequation{{\thesection}.{\arabic{equation}}}

\title[]{Absense of Positive Eigenvalues for the Linearized Elasticity System in the Half Space}
\author{ Zhiming Chen, Shiqi Zhou }
\address{LSEC, Institute of Computational Mathematics, Academy of
	Mathematics and Systems Science, Chinese Academy of Sciences,
	Beijing 100190, China}

\begin{abstract}
	In this paper, we prove the linearized elasticity system in the half-space with traction free boundry has no eigenvalues. We consider the constant Lame confficients and the desity outside the obstacle with different physical properties such as penetrable or non-penetrable, and for non-penetrable obstacles, the type of boundary conditions on the boundary of the obstacle.
\end{abstract}
\maketitle
\newcommand{\eps}{\varepsilon}
\newcommand{\RR}{\mathcal{R}}
\newtheorem{lem}{Lemma}[section]
\newtheorem{prop}{Proposition}[section]
\newtheorem{cor}{Corollary}[section]
\newtheorem{thm}{Theorem}[section]
\newtheorem{rem}{Remark}[section]
\newtheorem{alg}{Algorithm}[section]
\newtheorem{assum}{Assumption}[section]
\newtheorem{definition}{Definition}[section]


\newcounter{RomanNumber}
\newcommand{\MyRoman}[1]{\rm\setcounter{RomanNumber}{#1}\Roman{RomanNumber}}

\newcommand{\bL}{\mathbf{L}}
\newcommand{\bH}{\mathbf{H}}
\newcommand{\bW}{\mathbf{W}}
\newcommand{\bP}{\mathbf{P}}
\newcommand{\bQ}{\mathbf{Q}}
\newcommand{\bp}{\mathbf{p}}
\newcommand{\bq}{\mathbf{q}}
\newcommand{\uL}{u_{_{\rm L}}}
\newcommand{\vL}{v_{_{\rm L}}}
\newcommand{\tuL}{\tilde u_{_{\rm L}}}
\newcommand{\tvL}{\tilde v_{_{\rm L}}}
\newcommand{\fL}{f_{_{\rm L}}}
\newcommand{\gL}{g_{_{\rm L}}}
\newcommand{\bpL}{\bp_{_{\rm L}}}
\newcommand{\bqL}{\bq_{_{\rm L}}}
\newcommand{\tbpL}{\tilde{\bp}_{_{\rm L}}}
\newcommand{\tbqL}{\tilde{\bq}_{_{\rm L}}}
\newcommand{\tbpLf}{\tilde{\bp}_{_{\rm L,1}}}
\newcommand{\tbpLs}{\tilde{\bp}_{_{\rm L,2}}}
\newcommand{\tbqLf}{\tilde{\bq}_{_{\rm L,1}}}
\newcommand{\tbqLs}{\tilde{\bq}_{_{\rm L,2}}}
\newcommand{\bn}{\nu}
\newcommand{\bv}{\mathbf{v}}
\newcommand{\om}{\omega}
\newcommand{\pa}{\partial}
\newcommand{\la}{\langle}
\newcommand{\ra}{\rangle}
\newcommand{\lla}{\la{\hskip -2pt}\la}
\newcommand{\rra}{\ra{\hskip -2pt}\ra}
\newcommand{\jj}{\|{\hskip -0.8pt} |}
\newcommand{\al}{\alpha}
\newcommand{\ze}{\zeta}
\newcommand{\si}{\sigma}
\newcommand{\ep}{\varepsilon}
\newcommand{\na}{\nabla}
\newcommand{\vp}{\varphi}
\newcommand{\ga}{\gamma}
\newcommand{\Ga}{\Gamma}
\newcommand{\Om}{\Omega}
\newcommand{\de}{\delta}
\newcommand{\Th}{\Theta}
\newcommand{\De}{\Delta}
\newcommand{\Lam}{\Lambda}
\newcommand{\lam}{\lambda}
\newcommand{\tri}{\triangle}
\newcommand{\lj}{[{\hskip -2pt} [}
\newcommand{\rj}{]{\hskip -2pt} ]}
\newcommand{\bks}{\backslash}
%\newcommand{\diag}{\mathrm{diag}}
\newcommand{\diam}{\mathrm{diam}}
\newcommand{\osc}{\mathrm{osc}}
\newcommand{\meas}{\mathrm{meas}}
\newcommand{\dist}{\mathrm{dist}}

\newcommand{\mL}{\mathscr{L}}
\newcommand{\cT}{{\cal T}}
\newcommand{\cM}{{\cal M}}
\newcommand{\cE}{{\cal E}}
\newcommand{\cL}{{\cal L}}
\newcommand{\cF}{{\cal F}}
\newcommand{\cB}{{\cal B}}
\newcommand{\PML}{{\rm PML}}
\newcommand{\FEM}{{\rm FEM}}
\newcommand{\rd}{\,\mathrm{d}}

\renewcommand{\i}{\mathbf{i}}
\renewcommand{\v}{\mathbf{v}}
\renewcommand{\u}{\mathbf{u}}
\renewcommand{\r}{\mathbf{r}}
\newcommand{\R}{{\mathbb{R}}}
\newcommand{\A}{{\mathbb{A}}}
\newcommand{\Z}{{\mathbb{Z}}}
\newcommand{\C}{{\mathbb{C}}}
\newcommand{\I}{{\mathbb{I}}}
\renewcommand{\Re}{\mathrm{Re}\,}
\renewcommand{\Im}{\mathrm{Im}\,}
\renewcommand{\div}{\mathrm{div}}
\newcommand{\curl}{\mathrm{curl}}
\newcommand{\Curl}{\mathbf{curl}}

\newcommand{\Np}{\mathcal{N}_p}
\newcommand{\Ns}{\mathcal{N}_s}
\newcommand{\Tp}{\mathcal{T}_p}
\newcommand{\Ts}{\mathcal{T}_s}

%%%%%%%%%%%%%%%%%%%%%%%%%%%%%%%%%%%%%%%%%%%%%%%%%%%%%%%%%%%%%%%%%%%%
\newcommand{\be}{\begin{eqnarray}}
\newcommand{\ee}{\end{eqnarray}}
\newcommand{\ben}{\begin{eqnarray*}}
\newcommand{\een}{\end{eqnarray*}}
\newcommand{\nn}{\nonumber}

\section{Introduction}\label{section1}
In this paper, we consider the linearized and isotropic elasticity system defined on an unbounded domain $\Omega=\R_+^2\bks \bar{D}$ with traction free surface, where $D\subsetneq\R^2_+$ is a bounded Lipschitz domain with the unit outer normal $\nu$ to its boundary $\Gamma_D$.
We study the eigenvalues of the  following elastic scattering problem in the isotropic homogeneous medium half space with \emph{Lam\'{e}} constant $\lambda$ and $\mu$ and constant density $\rho\equiv1$:
\be\label{elastic_eq}
& &\nabla\cdot\sigma(\u) + \rho\omega^2 \u= f \ \ \ \ \mbox{in }\R_+^2\bks \bar{D}\\
& & \u=0 \ \ \mbox{on} \ \Ga_D  \ \ \mbox{and} \ \ \sigma(\u)\cdot e_2=0 \ \ \mbox{on} \ \Ga_0
\ee
together with the constitutive relation (Hookes law)
\ben
\sigma(\u) = 2\mu\ep(\u) + \lambda\div \u \I \\
\ep(\u)=\frac{1}{2}(\na \u +(\na \u)^T)
\een
where $\omega$ is the circular frequency, $\u(x_1,x_2)=(u_1(x),u_2(x))^T\in\C^2$ denotes the displacement fields and $\sigma(u)$ is the stress tensor. We also need to define the surface traction $T_x^n (\cdot)$ on the normal direction n,
\ben
T_x^n \u(x) := \sigma\cdot n = 2\mu\frac{\pa \u}{\pa n}+\lambda n\div \u + \mu n \times \curl \u
\een
For simplicity, let's introduce \emph{Lam\'{e}} operator $\Delta_e$ as
\ben
\Delta_e \u = (\lambda+2\mu)\nabla\nabla\cdot \u - \mu\nabla\times\nabla\times u=\nabla\cdot\sigma(\u)
\een
  For the elasticity system, the study of eigenvalue is little. To our knowledges. 
	The layout of the paper is as follows. In section 2 

\section{Absence of Positive Eigenvalues}
Throughout the paper, we will assume that for $z\in\mathbb{C}$, $z^{1/2}$ is the analytic branch of $\sqrt{z}$ such that $\Im (z^{1/2})\geq0$. This corresponds to the rigt half real axis as the branch cut in the complex plane. For $z=z_1+\i z_2,z_1,z_2\in\mathbb{R}$, we have
\be \label{convention_1}
z^{1/2}=sgn(z_2)\sqrt{\frac{|z|+z_1}{2}}+\i\sqrt{\frac{|z|-z_1}{2}}
\ee
For $z$ on the right half real axis, we take $z^{1/2}$ as the limit of $(z+\i\ep)^{1/2}$ as $\ep \to 0^+$.


By taking the Fourier transform of (\ref{pp1}-\ref{pp2}), we obtain ODEs for $x_2$ in $R_+$
\be
& & \mu \frac{d^2\hat u_1}{dx_2^2}+\i(\lambda+\mu)\xi\frac{d\hat u_2}{dx_2}+(\omega^2-(\lambda+2\mu)\xi^2)\hat u_1 =0 \label{pp3}\\
& & (\lambda+2 \mu)\frac{d^2\hat u_2}{dx_2^2}+\i(\lambda+\mu)\xi\frac{d\hat u_1}{dx_2}+(\omega^2-\mu \xi^2)\hat u_2 = 0 \label{pp4}
\ee
and the boundary coditions on $x_2=0$ are
\be
& & \mu\frac{d\hat u_1}{dx_2}+\i\mu\xi\hat u_2 = 0\label{pp5}\\
& & (\lambda+2\mu)\frac{d\hat u_2}{dx_2}+\i\lambda\xi\hat u_1 = 0 \label{pp6}
\ee
In order to work with real coefficient, we use the change of variables:
\ben
v_1=\i u_1 , \ \ \ \  v_2=u_2,  \ \ \ \  \v=(v_2,v_2)^T
\een
Then we have the following equations:
\ben
[\A_1 \frac{d^2}{dx_2^2} +(\A_2-(\A_2)^T)\xi\frac{d}{dx_2}-\A_3\xi^2+\omega^2]\v= \ \ \ \ \ \mbox{in} \ \  \R_+ \\
(\A_1 \frac{d}{dx_2} +\A_2)\v =0 \ \ \ \mbox{on} \ \ x_2=0
\een
where
\ben
\A_1=\Bigg( \begin{array}{cc}
\mu &  0\\
0 & \lambda+2\mu
\end{array} \Bigg) ,\ \  \ \ \
\A_2=\Bigg( \begin{array}{cc}
 0 & -\mu\\
\lambda & 0
\end{array} \Bigg), \ \ \ \
\A_3=\Bigg( \begin{array}{cc}
	\lambda+2\mu &  0\\
	0 & \mu
\end{array} \Bigg)
\een


\section*{References}
\bibliography{eee}
\end{document}
