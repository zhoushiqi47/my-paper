
\chapter{总结与展望}\label{chap:summary}

本文主要研究了半空间障碍物成像问题的逆时偏移算法。我们以一般半空间声波逆时偏移算法为出发点,首先提出了一种半空间无相位数据直接成像算法;然后我们考虑半空间分层介质模型下的散射问题和逆散射问题。特别地 我们关注于如何提出快速有效的数值重构算法,以确定嵌入在Pekeris开波导模型下障碍物的位置、大小和形状。通过对逆时偏移算法的研究,我们首先考虑点源确定问题,从点扩散函数出发,选取合适的反传播函数,最后提出了开波导逆时偏移算法以及该由算法的衍生出来的一些方法。数值算例表明这些算法继承了传统逆时偏移算法的优点,能够在不知道障碍物任何先验信息的条件下,对障碍物直接进行成像。

由于开波导模型的复杂性,开波导逆时偏移算法的分辨率分析较为复杂,很多理论还需要进行完善,接下来的工作可以从以下几个方面继续进行研究:
\begin{enumerate}
  \item 考虑第三章中开波导逆时偏移算法\ref{alg_wg}的分辨率分析,这需要寻求更为高级的数学工具对点扩散函数分析中的收敛性问题\ref{pro_convergence}给出一个合理的解释。
  \item 分析第四章中阻抗边界格林函数的估计,进而建立阻抗型开波导逆时偏移算法\ref{alg_imp}的分辨率分析。
  \item 建立半空间开波导任意层模型下障碍物成像问题的一般性数值重构算法。我们之前提到的算法都是需要计算反传播格林函数,而当半空间介质层数为三时,格林函数的表达式已经变得极其复杂,于是可以考虑是否能够建立一个更一般的框架,而去掉对计算格林函数的依赖性。
  \item 最后,本文仅考虑了声波开波导,而实际当中电磁波开波导以及弹性波开波导模型也具有非常多应用场景,可以尝试将本文提到的逆时偏移算法推广到电磁波和弹性波。
\end{enumerate}

除此之外,开波导格林函数的快速数值计算方法以及开波导正散射问题的高效数值求解算法也都是非常值得研究的课题。
